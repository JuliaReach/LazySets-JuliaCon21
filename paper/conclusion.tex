We conclude with a brief discussion of the past, present, and future development perspectives of LazySets.

\subsection{Origin of LazySets and current applications}

LazySets has its origins in a tool for reachability analysis of linear dynamical systems, using a compositional approach based on reducing high-dimensional lazy set representations into a sequence of low-dimensional projections that can be computed efficiently~\cite{BogomolovFFVPS18}.
%
This method presented the first approach to formally verify a $10{,}000$-dimensional benchmark from control engineering.
%
The reachability tool has since been rewritten in ReachabilityAnalysis.jl.
%
A preliminary exposition of these tools appeared in \cite{BogomolovFFPS19}.

\smallskip

We have recently applied LazySets to compute reachable states for linear wave propagation problems and heat transfer problems~\cite{forets2021combining}.
%
The scalability of the approach relies on exploiting the structure of linear systems through the support-function calculus and lazy evaluation.
%
Moreover, linear systems can be embedded in algorithms to analyze nonlinear systems \cite{forets2021reachability}.
%
Further case studies and comparison with other state-of-the-art tools can be found in \cite{ARCHCOMP20LINEAR, ARCHCOMP20NONLINEAR}.
%
LazySets has also been applied to the challenging domain of hybrid systems (systems with mixed discrete-continuous dynamics) for set propagation \cite{bogomolov2020reachability} and synthesis \cite{soto2021synthesis}.
%
Such problems require switching between different set representations and handling intersections efficiently and accurately.
%
Timed systems with non-deterministic events have been considered in \cite{forets2020efficient}; the approach is able to handle a large number of sets ($100$ million sets in zonotope representation), and it was shown to be an order of magnitude faster than competing tools~\cite{ARCHCOMP20LINEAR}.

\smallskip

Besides reachability analysis, LazySets can be used for other purposes. The tool has been adopted in a review article in the context of propagating sets through neural networks \cite{LiuALSBK21}, and new tools use LazySets for verification, e.g.,
\href{https://github.com/JuliaReach/NeuralNetworkAnalysis.jl}{NeuralNetworkAnalysis.jl}~\cite{schilling2021verification} and \href{https://github.com/sisl/OVERTVerify.jl}{OVERTVerify.jl}~\cite{sidrane2021overt}.
%
Other Julia packages using LazySets functionality include computation of invariant sets in \href{https://github.com/ueliwechsler/InvariantSets.jl}{InvariantSets.jl}, ray tracing for geometric optics in \href{https://github.com/microsoft/OpticSim.jl}{OpticSim.jl}, astronomical photometry in \href{https://github.com/JuliaAstro/Photometry.jl}{Photometry.jl}, thin film simulations in \href{https://github.com/Zitzeronion/Swalbe.jl}{Swalbe.jl}, and linear algebra with interval matrices in \href{https://github.com/JuliaIntervals/IntervalLinearAlgebra.jl}{IntervalLinearAlgebra.jl}.

\subsection{Related libraries}

There are few publicly available libraries with a similar aim as LazySets.
%
All these libraries are used in the context of reachability analysis.
%
HyPro is a C++ library for concrete representation and manipulation of sets such as convex polytopes and Taylor models and also offers a support-function representation of set operations \cite{SchuppAMK17}.
%
CORA is an actively developed Matlab library centered around zonotopes and contains implementations of zonotope bundles, matrix zonotopes, and polynomial zonotopes~\cite{Althoff15}.
%
The ellipsoidal toolbox is a Matlab library for ellipsoids \cite{kurzhanskiy2006ellipsoidal}.
%
SpaceEx is a C++ library that established the use of the support function in reachability analysis, but it is not open source \cite{frehse2011spaceex}.

\smallskip

Finally, we mention other related Julia packages with a different aim.
% 
As mentioned in Section~\ref{sec:applications}, \href{https://github.com/JuliaPolyhedra/Polyhedra.jl}{Polyhedra.jl} \cite{legat2021polyhedra} provides an interface for polyhedral computations and the double-description method; hence it complements nicely the lazy features available in our library.
%
\href{https://github.com/JuliaGeometry/GeometryBasics.jl}{GeometryBasics.jl} offers standard geometry types, creating a basis for graphics/plotting in Julia.
%
The more recent package
\href{https://github.com/JuliaGeometry/Meshes.jl}{Meshes.jl} is specialized on efficient and pure-Julia implementations of computational geometry and meshing algorithms.
%
Another Julia package for describing domains in Euclidean space is \href{https://github.com/JuliaApproximation/DomainSets.jl}{DomainSets.jl}; this package is being adopted for modeling PDE (partial differential equation) domains with \href{https://github.com/SciML/ModelingToolkit.jl}{ModelingToolkit.jl}.
%
Optimization is another scientific field where sets play a major role, with great contributions from Julia developers.
%
\href{https://github.com/jump-dev/MathOptInterface.jl}{MathOptInterface.jl}
\cite{legat2020mathoptinterface} is at the core of \href{https://github.com/jump-dev/JuMP.jl}{JuMP.jl}~\cite{DunningHuchetteLubin2017}, Julia's mainstream modeling language for mathematical optimization.
%
Applications include set programming in \href{https://github.com/blegat/SetProg.jl}{SetProg.jl}
and set distances in \href{https://github.com/jump-dev/MathOptSetDistances.jl}{MathOptSetDistances.jl}.

\subsection{LazySets in numbers}

At the time of writing (LazySets-v1.52.1), the package consists of 125 source files with almost 26k lines of code (LOC), 66 test files with over 5k LOC, and 66 documentation files (markdown) with over 4k LOC.
%
It is maintained by the authors of this article\footnote{\texttt{@mforets} and \texttt{@schillic} handles on \href{https://github.com/}{github.com}, respectively.}.
%
The project has received contributions from 13 other people.
%
LazySets was used in 19 research articles.


\subsection{Future work}

Set computations often do not allow for typical tricks you would expect to see in a Julia package. For instance, when working with generic polyhedra, there is very little structure, so most information cannot be statically inferred and needs to be computed from the concrete values (such as whether the polyhedron is empty). That is why there are so many set types: to bring in more structure for algorithms and dispatch.
%
Another challenge in set computations is to preserve type stability: in some cases, the output set type cannot be predicted in advance.

\smallskip

While many algorithms are already optimized, some functions still use a suboptimal, generic fallback.
%
We are interested to identify and fix such cases.
%
In our experience, one can often obtain speed-ups within several orders of magnitude by adding new methods.

\smallskip

Another direction is the use of trait-based dispatch, which may be useful as a workaround for limitations of the Julia type system, e.g., that it does not allow for multiple inheritance.
%
Expressing properties of sets that fall outside the established LazySets type hierarchy would allow for even further flexibility.

\smallskip

Our next aimed milestone is proper support of non-convex set representations. While functionality to operate with such set representations is already available, the interoperability between convex and non-convex sets has room for improvement.

\smallskip

Finally, LazySets has a solid documentation of its API by an extensive use of docstrings and uses \href{https://github.com/JuliaDocs/Documenter.jl}{Documenter.jl} for its online documentation.
%
We plan to add more introductory examples and tutorials for first-time users.
